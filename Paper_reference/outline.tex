\section{Algorithm Outline}
\label{sec:outline}



Our high-level framework is similar to previous works \cite{duan2023, VXXZ24}, and we borrow definitions such as interface tensors, compatibility, and usefulness. 
We begin by defining the notion of interface tensors, which captures the structure of the intermediate tensors we obtain from different stages of our algorithm and pass to the next stages.

\begin{definition}[Interface Tensor]\label{def:interface-tensor}
  For a positive integer $\lvl\ge 1$ and any constant $0\le \eps \le 1$, a level-$\lvl$ $\eps$-\emph{interface tensor} $\T^*$ with parameter list 
  \[\{(n_t, i_t, j_t, k_t, \splresXt, \splresYt, \splresZt)\}_{t \in [s]}\]
  is defined as
  \[
    \T^* \defeq \bigotimes_{t = 1}^{s} T_{i_t, j_t, k_t}^{\otimes n_t}[\splresXt, \splresYt, \splresZt, \eps],
  \]
  where $i_t+j_t+k_t = 2^{\lvl}$ for every $t\in [s]$ (i.e., $T_{i_t, j_t, k_t}$ is a level-$\lvl$ constituent tensor) and  $\splresXt, \splresYt, \splresZt$ are level-$\lvl$ complete split distributions for $X$-, $Y$-, $Z$-variables respectively. We call each $T_{i_t, j_t, k_t}^{\otimes n_t}[\splresXt, \splresYt, \splresZt, \eps]$ a \emph{term} of $\T^*$. When $\eps = 0$, we will simply call $\T^*$ a level-$\lvl$ interface tensor. 
\end{definition}

Our algorithm takes $\bigbk{\CW_q^{\otimes 2^{\lvl^*-1}}}^{\otimes n}$ as input, and applies the global stage of the algorithm described in \cref{sec:global} to degenerate it into independent copies of a  level-$\lvl^*$ $\eps_{\lvl^*}$-interface tensor. For every $\lvl \in [\lvl^*]$, we will have a parameter $\eps_\lvl$ controlling the error margin of level-$\lvl$ interface tensors. Then for $\lvl = \lvl^*, \lvl^* \!-\! 1, \ldots, 2$, we apply the constituent stage of the algorithm described in \cref{sec:constituent} to degenerate a level-$\lvl$ $\eps_\lvl$-interface tensor into a tensor product between independent copies of level-$(\lvl-1)$ $\eps_{\lvl-1}$-interface tensors and a matrix multiplication tensor. Eventually, we obtain a tensor product between independent copies of level-$1$ $\eps_1$-interface tensors and a matrix multiplication tensor, which can then be degenerated into independent copies of matrix multiplication tensors. Thus, the overall algorithm degenerates $\bigbk{\CW_q^{\otimes 2^{\lvl^*-1}}}^{\otimes n}$ into independent copies of matrix multiplication tensors as desired. 







\subsection{Algorithm Outline}\label{subsec:outline}

The framework for the constituent stage is similar to the framework for the global stage.
For simplicity, we only present the outline of our global stage algorithm for $\eps = 0$ as it captures the main ideas of our algorithm and illustrates the main differences between our algorithm and \cite{VXXZ24}'s. A high-level comparison between our algorithm and \cite{VXXZ24}'s is provided in \cref{fig:outline-compare}. 

\begin{figure}
    \centering
    \tikzset{>={Latex[width=1.5mm,length=1mm]},
 base/.style = {rectangle, rounded corners, draw=black, minimum width=11cm, minimum height=1cm, text centered},
 regularblock/.style = {base},
 small/.style = {rectangle, rounded corners, draw=black, minimum width=3.25cm, minimum height=1.5cm, text centered},
 big/.style = {rectangle, rounded corners, draw=black, minimum width=11cm, minimum height=2cm, text centered}
}


\begin{tikzpicture}[scale=0.7,every node/.style={scale=0.7}]


\node (input) at (0,0) [regularblock] {\textbf{Input}: $(\CW_q^{\otimes 2^{\ell-1}})^{\otimes n}$};

\node (alpha) at (0,-2) [regularblock] {Zero out level-$\ell$ blocks according to $\alpha$};

\node (hash) at (0,-4) [regularblock] {Asymmetric hashing};


\node[align=left] (Zcomp1) at (-3.875, -7) [small] {$Z$-Compatibility\\ Zero out I};

\node[align=left] (Zcomp2) at (0, -7) [small] {$Z$-Compatibility\\ Zero out II};

\node[align=left] (Zcomp3) at (3.875, -7) [small] {$Z$-Usefulness\\ Zero out};

\node (fix) at (0,-9.75) [regularblock] {Fix holes};

\node (out) at (0,-11.75) [regularblock] {\textbf{Output}: Independent copies of interface tensors};

\node[scale=1.3] at (0,-13) {(a) Global stage algorithm outline in \cite{VXXZ24}};

\draw[->] (input) -- (alpha);
\draw[->] (alpha) -- (hash);
\draw[->] (Zcomp1) -- (Zcomp2);
\draw[->] (Zcomp2) -- (Zcomp3);
\draw[->] (fix) -- (out);


\draw[->] (5.5,-4) -- (5.75, -4) -- (5.75, -5.5) -- (-5.75, -5.5) -- (-5.75, -7) -- (-5.5, -7);
\draw[->] (5.5,-7) -- (5.75, -7) -- (5.75, -8.5) -- (-5.75, -8.5) -- (-5.75,-9.75) -- (-5.5, -9.75);

\node[blue] at (0,-5.25) {Every level-$\ell$ $X$-block and $Y$-block is in a unique level-$\ell$ triple};
\node[blue] at (0,-8.25) {Every level-$1$ $Z$-block is in a unique level-$\ell$ triple};


\coordinate (shift1) at (12.5,0);
\begin{scope}[shift=(shift1)]

\node (input) at (0,0) [regularblock] {\textbf{Input}: $(\CW_q^{\otimes 2^{\ell-1}})^{\otimes n}$};

\node (alpha) at (0,-1.5) [regularblock] {Zero out level-$\ell$ blocks according to $\alpha$};

\node (hash) at (0,-3) [regularblock,fill=green!30] {More asymmetric hashing};


\node[align=left,fill=green!30] (Ycomp1) at (-3.875, -5.25) [small] {$Y$-Compatibility\\ Zero out I};

\node[align=left,fill=green!30] (Ycomp2) at (0, -5.25) [small] {$Y$-Compatibility\\ Zero out II};

\node[align=left,fill=green!30] (Ycomp3) at (3.875, -5.25) [small] {$Y$-Usefulness\\ Zero out};


\node[align=left] (Zcomp1) at (-3.875, -7.75) [small] {$Z$-Compatibility\\ Zero out I};

\node[align=left] (Zcomp2) at (0, -7.75) [small] {$Z$-Compatibility\\ Zero out II};

\node[align=left] (Zcomp3) at (3.875, -7.75) [small] {$Z$-Usefulness\\ Zero out};

\node (fix) at (0,-10.25) [regularblock] {Fix holes};

\node (out) at (0,-11.75) [regularblock] {\textbf{Output}: Independent copies of interface tensors};

\draw[->] (input) -- (alpha);
\draw[->] (alpha) -- (hash);
\draw[->] (Ycomp1) -- (Ycomp2);
\draw[->] (Ycomp2) -- (Ycomp3);
\draw[->] (Zcomp1) -- (Zcomp2);
\draw[->] (Zcomp2) -- (Zcomp3);
\draw[->] (fix) -- (out);

\node[blue] at (0,-4) {Every level-$\ell$ $X$-block is in a unique level-$\ell$ triple};

\draw[->] (5.5,-3) -- (5.75, -3) -- (5.75, -4.25) -- (-5.75, -4.25) -- (-5.75, -5.25) -- (-5.5, -5.25);

\node[blue] at (0,-6.5) {Every level-$1$ $Y$-block is in a unique level-$\ell$ triple};

\draw[->] (5.5,-5.25) -- (5.75, -5.25) -- (5.75, -6.75) -- (-5.75, -6.75) -- (-5.75, -7.75) -- (-5.5, -7.75);

\node[blue] at (0,-9) {Every level-$1$ $Z$-block is in a unique level-$\ell$ triple};

\draw[->] (5.5,-7.75) -- (5.75, -7.75) -- (5.75, -9.25) -- (-5.75, -9.25) -- (-5.75,-10.25) -- (-5.5, -10.25);

\node[scale=1.3] at (0,-13) {(b) Global stage algorithm outline in this work};
\end{scope}
    
\end{tikzpicture}

    \caption{Main differences between the global stage algorithm in \cite{VXXZ24} and this work as outlined in \cref{subsec:outline}. Other technical differences are omitted.}
    \label{fig:outline-compare}
\end{figure}

As mentioned earlier, the goal of the global stage is to take $\bigbk{\CW_q^{\otimes 2^{\lvl-1}}}^{\otimes n}$ as input (for convenience, we will use $\lvl$ to denote $\lvl^*$), and output a set of level-$1$-independent level-$\lvl$ interface tensors. Since our algorithm treats the $X, Y, Z$ dimensions asymmetrically and there are six permutations of the $X, Y, Z$ dimensions, we can essentially apply our algorithm in six different ways. As a result, we will split $\bigbk{\CW_q^{\otimes 2^{\lvl-1}}}^{\otimes n}$ into $\bigotimes_{r \in [6]} \bigbk{\CW_q^{\otimes 2^{\lvl-1}}}^{\otimes A_r \cdot n}$ for some $A_1, \ldots, A_6 \ge 0$  and $\sum_{r = 1}^6 A_r = 1$, where we call each $\bigbk{\CW_q^{\otimes 2^{\lvl-1}}}^{\otimes A_r \cdot n}$ a \emph{region}. We apply our algorithm for each permutation of the $X, Y, Z$ dimensions on one of the regions. We note that in \cite{VXXZ24}, two of the three dimensions are treated symmetrically, so they only need to split the input tensor into $3$ regions instead of $6$. For simplicity, we outline each step of our algorithm on a fixed permutation without splitting $\bigbk{\CW_q^{\otimes 2^{\lvl-1}}}^{\otimes n}$ into regions in the following.



\begin{enumerate}
\item \textbf{Zero-out according to \boldmath$\alpha$.} This is a standard first step present in all previous applications of the laser method. For some distribution $\alpha$ over level-$\lvl$ constituent tensors, let $\alphx, \alphy, \alphz$ be the marginal distributions of $\alpha$ on the $X$-, $Y$-, $Z$-dimensions respectively. We zero out $X$-, $Y$-, $Z$-variable blocks that are not consistent with $\alphx, \alphy, \alphz$ respectively. 

\item \textbf{Asymmetric hashing.} Similar to \cite{duan2023, VXXZ24}, we then apply asymmetric hashing. Unlike previous works, where after the asymmetric hashing each level-$\lvl$ $X$-variable block $X_I$ or $Y$-variable block $Y_J$ is contained in a unique level-$\lvl$ triple $X_I Y_J Z_K$, we only require each level-$\lvl$ $X$-block to be contained in a unique triple. Moreover, the number of triples containing each level-$\lvl$ $Y$-block is at most the number of triples containing each level-$\lvl$ $Z$-block. 


\item \textbf{\boldmath$Y$-compatibility zero-outs.} In \cite{duan2023, VXXZ24}, the next step is to do some zeroing outs based on a notion of compatibility defined for level-$1$ $Z$-blocks, provided that every level-$1$ $X$- and $Y$-block is contained in a unique block triple. So in our algorithm, we first need to ensure that each level-$1$ $Y$-block is contained in a unique block triple by performing the following sub-steps. 

Note that our goal is to obtain an interface tensor with  $\left\{\splres_{\itX, i, j, k}, \splres_{\itY, i, j, k}, \splres_{\itZ, i, j, k}\right\}_{i+j+k = 2^\lvl}$ as its complete split distributions. Since each level-$\lvl$ $X$-block is contained in a unique block triple, we are already able to identify the subset of indices $t$ with $X_t = i$, $Y_t = j$, $Z_t = k$ for every $i, j, k$. This means that we can compute the complete split distributions for each level-$1$ $X$-block with respect to these indices, and zero out any level-$1$ $X$-block that is not consistent with $\left\{\splres_{\itX, i, j, k}\right\}_{i+j+k = 2^\lvl}$. 

\begin{enumerate}
    \item \textbf{\boldmath$Y$-compatibility zero-out I.} Then we define a notion of compatibility for a level-$1$ $Y$-block with respect to a level-$\lvl$ block triple containing it. We say a level-$1$ $Y$-block is compatible with a level-$\lvl$ block triple if they satisfy the compatibility conditions. We zero out all level-$1$ $Y$-blocks that are not compatible with any triple. 
    \item \textbf{\boldmath$Y$-compatibility zero-out II.} In this step, we zero out the level-$1$ $Y$-blocks that are compatible with more than one triple. After this step, we can ensure that each level-$1$ $Y$-block is contained in a unique level-$\lvl$ block triple and it is compatible with that triple. 
    \item \textbf{\boldmath$Y$-usefulness zero-out.} At this point, each level-$1$ $Y$-block is contained in a unique triple, so we can zero out a level-$1$ $Y$-block if it is not consistent with $\left\{\splres_{\itY, i, j, k}\right\}_{i+j+k = 2^\lvl}$, similar to what we did to the level-$1$ $X$-blocks. 
\end{enumerate}


\item \textbf{\boldmath$Z$-compatibility zero-outs.} Although after the more asymmetric hashing step, we only have $X$-blocks in a unique triple, notice that after $Y$-compatibility zero-outs, we achieve the property that now every level-$1$ $X$- and $Y$-block is compatible with a unique level-$\ell$ triple. This property is weaker than the property achieved in \cite{duan2023, VXXZ24} after their original asymmetric hashing steps, but this is sufficient for later steps. In particular, in our $Z$-compatibility zero-out, we can perform essentially the same procedure as in \cite{VXXZ24} to achieve the property that every level-$1$ $Z$-block is compatible with a unique level-$\lvl$ triple.

\item  \textbf{Fixing holes.} Now every level-$1$ block is contained in a unique level-$\ell$ triple, but some of the level-$1$ blocks we zeroed out in $Y$-compatibility zero-outs and $Z$-compatibility zero-outs become holes, i.e., the copies of the interface tensors we obtain miss some variables due to the compatibility zero-outs. We use the following result to fix these holes.

\begin{theorem}[{\cite[Corollary 3.2]{VXXZ24}}]\label{thm:fix-holes}
    Let $T$ be a level-$\ell$ interface tensor with parameter list
    \[\{(n_t, i_t, j_t, k_t, \splresXt, \splresYt, \splresZt)\}_{t \in [s]}.\]
    Let $N = 2^{\ell-1} \cdot \sum_{t\in [s]} n_t$. Suppose $T_1, \dots, T_r$ are broken copies of $T$ where $\le \frac{1}{8N}$ fraction of level-1 $X$, $Y$, and $Z$-blocks are holes. If $r \ge 2^{C_1 N / \log N}$ for some large enough constant $C_1>0$, the direct sum $\bigoplus_{i = 1}^r T_i$ can degenerate into an unbroken copy of $T$.
\end{theorem}



\end{enumerate}
